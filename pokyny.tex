\chapter{Úvod}
Výsledná struktura vaší práce a názvy a rozsahy jednotlivých kapitol se samozřejmě budou lišit podle typu práce a podle konkrétní povahy zpracovávaného tématu. 

\section{Jak používat tuto šablonu}

\chapter{Pokyny a návody k formátování textu práce}
Používat se dají všechny příkazy systému \LaTeX. Existuje velké množství volně přístupné dokumentace, tutoriálů, příruček a dalších materiálů v elektronické podobě. Výchozím bodem, kromě Googlu, může být stránka CSTUG (Czech Tech Users Group) \cite{CSTUG}. Tam najdete odkazy na další materiály.  Vetšinou dostačující a přehledně organizovanou elektronikou dokumentaci najdete například na \cite{latexdocweb} nebo \cite{latexwiki}.

Existují i různé nadstavby nad systémy \TeX{} a \LaTeX, které výrazně usnadní psaní textu zejména začátečníkům. Z mnoha možných uvádíme: Kile\footnote{\url{http://kile.sourceforge.net/}}, TexMaker\footnote{\url{http://www.xm1math.net/texmaker/}}, LyX\footnote{\url{http://www.lyx.org/}}.


%%%%%%%%%%%%%%%%%%%%%%%%%% 
% Pro snadnejsi praci s vetsimi texty je rozumne tyto rozdelit
% do samostatnych souboru nejlepe dle kapitol a tyto potom vkladat
% pomoci prikazu \include{jmeno_souboru}.
% Napr.:
% \include{1_uvod}
% \include{2_teorie}
% atd...
%%%%%%%%%%%%%%%%%%%%%%%%%% 



\section{Tabulky}
Existuje více způsobů, jak sázet tabulky. Například je možno použít prostředí \verb|table|, které je velmi podobné prostředí \verb|figure|. 

\begin{table}
\begin{center}
\begin{tabular}{|c|l|l|}
\hline
\textbf{DTD} & \textbf{construction} & \textbf{elimination} \\
\hline
$\mid$ & \verb+in1|A|B a:sum A B+ & \verb+case([_:A]a)([_:B]a)ab:A+\\
&\verb+in1|A|B b:sum A B+ & \verb+case([_:A]b)([_:B]b)ba:B+\\
\hline
$+$&\verb+do_reg:A -> reg A+&\verb+undo_reg:reg A -> A+\\
\hline
$*,?$& the same like $\mid$ and $+$ & the same like $\mid$ and $+$\\
& with \verb+emtpy_el:empty+ & with \verb+emtpy_el:empty+\\
\hline
R(a,b) & \verb+make_R:A->B->R+ & \verb+a: R -> A+\\
 & & \verb+b: R -> B+\\
\hline
\end{tabular}
\end{center}
\caption{Ukázka tabulky}
\label{tab:tab1}
\end{table}

Zdrojový text tabulky \ref{tab:tab1} vypadá takto:
\begin{verbatim}
\begin{table}
\begin{center}
\begin{tabular}{|c|l|l|}
\hline
\textbf{DTD} & \textbf{construction} & \textbf{elimination} \\
\hline
$\mid$ & \verb+in1|A|B a:sum A B+ & \verb+case([_:A]a)([_:B]a)ab:A+\\
&\verb+in1|A|B b:sum A B+ & \verb+case([_:A]b)([_:B]b)ba:B+\\
\hline
$+$&\verb+do_reg:A -> reg A+&\verb+undo_reg:reg A -> A+\\
\hline
$*,?$& the same like $\mid$ and $+$ & the same like $\mid$ and $+$\\
& with \verb+emtpy_el:empty+ & with \verb+emtpy_el:empty+\\
\hline
R(a,b) & \verb+make_R:A->B->R+ & \verb+a: R -> A+\\
 & & \verb+b: R -> B+\\
\hline
\end{tabular}
\end{center}
\caption{Ukázka tabulky}
\label{tab:tab1}
\end{table}
\begin{table}
\end{verbatim}

A pokud máte svá data v CSV můžete použít některou z knihoven nabízených v   http://texblog.org/2012/05/30/generate-latex-tables-from-csv-files-excel/ 

\section{Odkazy v textu}
\subsection{Odkazy na literaturu}
Jsou realizovány příkazem \verb|\cite{odkaz}|. 

Seznam literatury je dobré zapsat do samostatného souboru a ten pak zpracovat programem bibtex (viz soubor \verb|reference.bib|). Zdrojový soubor pro \verb|bibtex| vypadá například takto:
\begin{verbatim}
@Article{Chen01,
  author  = "Yong-Sheng Chen and Yi-Ping Hung and Chiou-Shann Fuh",
  title   = "Fast Block Matching Algorithm Based on 
             the Winner-Update Strategy",
  journal = "IEEE Transactions On Image Processing",
  pages   = "1212--1222",
  volume  =  10,
  number  =   8,
  year    = 2001,
}

@Misc{latexdocweb,
  author  = "",
  title   = "{\LaTeX} --- online manuál",
  note    = "\verb|http://www.cstug.cz/latex/lm/frames.html|",
  year    = "",
}
...
\end{verbatim}

%11.12.2008, 3.5.2009
\textbf{Pozor:} Sazba názvů odkazů je dána Bib\TeX{} stylem\\ (\verb|\bibliographystyle{abbrv}|). 
%Budete-li používat české prostředí (\verb|\usepackage[czech]{babel}|), 
Bib\TeX{} tedy obvykle vysází velké pouze počáteční písmeno z názvu zdroje, 
ostatní písmena zůstanou malá bez ohledu na to, jak je napíšete. 
Přesněji řečeno, styl může zvolit pro každý typ publikace jiné konverze. 
Pro časopisecké články třeba výše uvedené, jiné pro monografie (u nich často bývá 
naopak velikost písmen zachována).

Pokud chcete Bib\TeX u napovědět, která písmena nechat bez konverzí 
(viz \texttt{title = "\{$\backslash$LaTeX\} -{}-{}- online manuál"} 
v~předchozím příkladu), je nutné příslušné písmeno (zde celé makro) uzavřít 
do složených závorek. Pro přehlednost je proto vhodné celé parametry 
uzavírat do uvozovek (\texttt{author = "\dots"}), nikoliv do složených závorek.

Odkazy na literaturu ve zdrojovém textu se pak zapisují:
\begin{verbatim}
Podívejte se na \cite{Chen01}, 
další detaily najdete na \cite{latexdocweb}
\end{verbatim}

Vazbu mezi soubory \verb|*.tex| a \verb|*.bib| zajistíte příkazem 
\verb|\bibliography{}| v souboru \verb|*.tex|.  V našem případě tedy zdrojový 
dokument \verb|thesis.tex| obsahuje příkaz\\
\verb|\bibliography{reference}|.

Zpracování zdrojového textu s odkazy se provede postupným voláním programů\\
\verb|pdflatex <soubor>| (případně \verb|latex <soubor>|), \verb|bibtex <soubor>| 
a opět\\ \verb|pdflatex <soubor>|.\footnote{První volání \texttt{pdflatex} 
vytvoří soubor s~koncovkou \texttt{*.aux}, který je vstupem pro program 
\texttt{bibtex}, pak je potřeba znovu zavolat program \texttt{pdflatex} 
(\texttt{latex}), který tentokrát zpracuje soubory s příponami \texttt{.aux} a 
\texttt{.tex}. 
Informaci o případných nevyřešených odkazech (cross-reference) vidíte přímo při 
zpracovávání zdrojového souboru příkazem \texttt{pdflatex}. Program \texttt{pdflatex} 
(\texttt{latex}) lze volat vícekrát, pokud stále vidíte nevyřešené závislosti.}


Níže uvedený příklad je převzat z dříve existujících pokynů studentům, kteří 
dělají svou diplomovou nebo bakalářskou práci v~Grafické skupině.\footnote{Několikrát 
jsem byl upozorněn, že web s těmito pokyny byl zrušen, proto jej zde přímo necituji. 
Nicméně příklad sám o sobě dokumentuje obecně přijímaný konsensus ohledně citací 
v~bakalářských a diplomových pracích na KP.} Zde se praví:
\begin{small}
\begin{verbatim}
...
j) Seznam literatury a dalších použitých pramenů, odkazy na WWW stránky, ...
 Pozor na to, že na veškeré uvedené prameny se musíte v textu práce 
 odkazovat -- [1]. 
Pramen, na který neodkazujete, vypadá, že jste ho vlastně nepotřebovali 
a je uveden jen do počtu. Příklad citace knihy [1], článku v časopise [2], 
stati ve sborníku [3] a html odkazu [4]: 
[1] J. Žára, B. Beneš;, and P. Felkel. 
     Moderní počítačová grafika. Computer Press s.r.o, Brno, 1 edition, 1998. 
     (in Czech). 
[2] P. Slavík. Grammars and Rewriting Systems as Models for Graphical User 
     Interfaces. Cognitive Systems, 4(4--3):381--399, 1997. 
[3] M. Haindl, Š. Kment, and P. Slavík. Virtual Information Systems. 
     In WSCG'2000 -- Short communication papers, pages 22--27, Pilsen, 2000. 
     University of West Bohemia. 
[4] Knihovna grafické skupiny katedry počítačů: 
     http://www.cgg.cvut.cz/Bib/library/ 
\end{verbatim}
\end{small}
\ldots{} abychom výše citované odkazy skutečně našli v (automaticky generovaném) seznamu literatury tohoto textu, musíme je nyní alespoň jednou citovat: Kniha \cite{kniha}, článek v~časopisu \cite{clanek}, příspěvek na konferenci \cite{sbornik}, www odkaz \cite{www}.

Ještě přidáme další ukázku citací online zdrojů podle české normy. Odkaz na wiki o frameworcich \cite{wiki:framework} a ORM \cite{wiki:orm}. Použití viz soubor \verb|reference.bib|. V seznamu literatury by nyní měly být živé odkazy na zdroje. V \verb|reference.bib| je zcela nový typ publikace. Detaily dohledal a dodal Petr Dlouhý v dubnu 2010. Podrobnosti najdete ve zdrojovém souboru tohoto textu v komentáři u příkazu \verb|\thebibliography|.

\subsection{Odkazy na obrázky, tabulky a kapitoly}
\begin{itemize}
\item Označení místa v textu, na které chcete později čtenáře práce odkázat, se provede příkazem \verb|\label{navesti}|. Lze použít v prostředích \verb|figure| a  \verb|table|, ale též za názvem kapitoly nebo podkapitoly.
\item Na návěští se odkážeme příkazem \verb|\ref{navesti}| nebo \verb|\pageref{navesti}|.
\end{itemize}

\section{Rovnice, centrovaná, číslovaná matematika}
Jednoduchý matematický výraz zapsaný přímo do textu se vysází pomocí prostředí \verb|math|, resp. zkrácený zápis pomocí uzavření textu rovnice mezi znaky \verb|$|.

Kód \verb|$ S = \pi * r^2 $| bude vysázen takto: $ S = \pi * r^2 $.

Pokud chcete nečíslované rovnice, ale umístěné centrovaně na samostatné řádky, pak lze použít prostředí \verb|displaymath|, resp. zkrácený zápis pomocí uzavření textu rovnice mezi znaky \verb|$$|. Zdrojový kód: 
\begin{verb}
|$$ S = \pi * r^2 $$|
\end{verb}
bude pak vysázen takto:
$$ S = \pi * r^2 $$

Chcete-li mít rovnice číslované, je třeba použít prostředí \verb|eqation|. Kód:
\begin{verbatim}
\begin{equation}
  S = \pi * r^2
\end{equation}

\begin{equation}
  V = \pi * r^3
\end{equation}
\end{verbatim}
je potom vysázen takto:
\begin{equation}
  S = \pi * r^2
\end{equation}

\begin{equation}
  V = \pi * r^3
\end{equation}

\section{Kódy a algoritmy}
\subsection{Zdrojové kódy}
Chceme-li vysázet například část zdrojového kódu programu, hodí se prostředí \verb|verbatim|, které je bez formátování: 
\begin{verbatim}
         (* nickname2 *)
Lego> Refine in1
             (do_reg (nickname1 h));
Refine by  in1 (do_reg (nickname1 h))
   ?4 : pcdata
   ?5 : pcdata
          (* surname2 *)
Lego> Refine surname1 h;
Refine by  surname1 h
   ?5 : pcdata
          (* email2 *)
Lego> Refine undo_reg (email1 h);
Refine by  undo_reg (email1 h)
*** QED ***
\end{verbatim}

nebo se dá použít \verb|listings|, což je package, který umožňuje i syntax higlighting podle jazyka:

\lstset{language=Python} %následuje python kód a použije se python syntax
\begin{lstlisting}
	print 'Hello, world!'
\end{lstlisting}

a umožňuje načíst i přiložené zdrojové soubory:

%\lstinputlisting[language=C]{code/hello.c}

\subsection{Algoritmy}
Pokud chcete v práci popsat obecné algoritmy s využitím pseudokódu, můžete použít knihovny \verb|algorithmicx| a \verb|algpseudocode|:
\begin{algorithmic}
\If {$i\geq maxval$}
    \State $i\gets 0$
\Else
    \If {$i+k\leq maxval$}
        \State $i\gets i+k$
    \EndIf
\EndIf
\end{algorithmic}

\section{Zkratky}
V tomto textu používám několik zkratek, třeba 2D\nomenclature{2D}{Two-Dimensional} nebo \nomExpl{KNB}{Killing nanobots}. V úvodní části dokumentu můžete nastavit/upravit příkaz pro zadání zkratky, pokud např. chcete mít význam zkratky pod čarou. Všechny zkratky se vytisknou podle abecedy (\nom{ABC}{Zkratka pro abecedu}) v příloze \ref{apx:zkratky}, aby toto fungovalo musíte vybudovat index pomocí příkazu:
\begin{verbatim}
		makeindex soubor.nlo -s nomencl.ist -o soubor.nls
\end{verbatim}

\section{České uvozovky}
V souboru \verb|k336_thesis_macros.tex| je příkaz \verb|\uv{}| pro sázení českých uvozovek. \uv{Text uzavřený do českých uvozovek.}

\chapter{Závěr}

\begin{itemize}
\item Zhodnocení splnění cílů DP/BP a  vlastního přínosu práce (při formulaci je třeba vzít v potaz zadání práce).
\item Diskuse dalšího možného pokračování práce.
\end{itemize} 

