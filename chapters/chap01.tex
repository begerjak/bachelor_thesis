\addbibresource{reference.bib}

\chapter{Úvod}\label{chap01}
Tato bakalářská práce se zabývá návrhem a implementací software pro ovládání a kalibraci sítě hybridních částicových pixelových detektorů umístěných uvnitř experimentu ATLAS na LHC v CERN - projekt AtlasTPX. 
Jelikož proces kalibrace je zcela nezávislý na následném řízné činnosti těchto detektorů, je tento software členěn na dvě nezávislé části, a to na energetickou kalibraci částicových pixelových detektorů (viz kapitola \ref{calib}) a řízení sítě těchto detektorů - ATLAS TPX (viz kapitola \ref{atlas}).

\todo

\section{Motivace}
Ionizující záření je spjato s naším světem už od počátku jeho existence. Jeho studium začalo koncem 19. století a pomáhá nám pochopit podstatu hmoty, její interakce s prostředím a další vlastnosti. Tyto poznatky našly své uplatnění v mnoha oborech, jako například v defektoskopii, zdravotnictví, energetice a v mnoha dalších. Spolu s rostoucími poznatky o ionizujících záření a s technických pokrokem se rozvíjela i detekční technika, která za poslední století prodělala veliký posun. Od prvních bublinových komor, až po nejmodernější polovodičové pixelové detektory, kterými se tato práce zabývá. 


%V rámci svého působení v Ústavu technické a experimentální fyziky ČVUT v Praze (který je členem kolaborace Medipix v CERN) jsem se podílel na vývoji software pro hybridní pixelové detektory rodiny  Medipix
%Hybridních pixelové detektory rodiny Medipix a Timepix pomocí matice $256*256$ pixelů 
% díky vysoké míře integrace se na ploše okolo x $cm^2$ nachází $65 536$ aktivních pixelů. 
	