\addbibresource{reference.bib}

\chapter{Úvod}\label{chap01}
Tato bakalářská práce se zabývá návrhem a~implementací software pro ovládání a~energetickou kalibraci sítě hybridních částicových pixelových detektorů umístěných uvnitř experimentu ATLAS na LHC v~CERN - projekt ATLAS TPX. 

Jelikož proces kalibrace je zcela nezávislý na následném řízné činnosti těchto detektorů, je tento software členěn na dvě nezávislé části, a~to na energetickou kalibraci částicových pixelových detektorů (viz kapitola \ref{calib}) a~řízení sítě těchto detektorů - ATLAS TPX (viz kapitola~\ref{atlas}).

Kalibrační software bude uživateli umožňovat průchod procesem zpracování kalibračních dat od sestavení spekter z~naměřených snímků, přes jejich analýzu a~vytvoření jednotlivých kalibračních bodů, až po sestavení kalibrační funkce pro jednotlivé pixely detektoru.

Řídící software sítě ATLAS TPX pak zase bude sloužit pro nezávislé ovládání funkce těchto detektorů. Bude umožňovat nastavování různých parametrů detektorů, řízení akvizice snímků, vyčítání naměřených dat a~jejich ukládání, ev. jejich odesílání datovému serveru pro jejich hlubší analýzu a~uložení do perzistentní centrální CERNské databáze (tzv. \texttt{EOS}). Zároveň tento řídící software bude poskytovat \texttt{JSON REST API} pro své vzdálené řízení a~také pro poskytování stavových informací o teto sítí CERNu.

\section{Motivace}
Ionizující záření je spjato s~naším světem už od počátku jeho existence. Jeho studium začalo koncem 19. století a~pomáhá nám pochopit podstatu hmoty, její interakce s~prostředím a~další vlastnosti. Tyto poznatky našly své uplatnění v~mnoha oborech, jako například v~defektoskopii, zdravotnictví, energetice a~v mnoha dalších. Spolu s~rostoucími poznatky o ionizujících záření a~s technických pokrokem se rozvíjela i detekční technika, která za poslední století prodělala veliký posun. Od prvních bublinových komor, až po nejmodernější polovodičové pixelové detektory, kterými se tato práce zabývá. 


%V rámci svého působení v~Ústavu technické a~experimentální fyziky ČVUT v~Praze (který je členem kolaborace Medipix v~CERN) jsem se podílel na vývoji software pro hybridní pixelové detektory rodiny  Medipix

	
\section{Struktura dokumentu}
\begin{description}
	\item[Kapitola] \ref{det} - \nameref{det}:
		V této kapitole budou představeny hybridní částicové pixelové detektory rodiny Medipix, jejich rozdělení, principy detekce, módy a~další relevantní vlastnosti k~této práci.
	\item[Kapitola] \ref{calib} - \nameref{calib}:
		Tato kapitola bude pojednávat o metodách energetické kalibrace částicových pixelových detektorů rodiny Medipix, pracujících v~\texttt{Time-Over-Treshold} módu a~také zde bude zmíněna implementace jedné z~těchto metod pro účely kalibrace detektorové sítě ATLAS TPX.
	\item[Kapitola] \ref{atlas} - \nameref{atlas}:
		V rámci této kapitoly bude popsán návrh softwarové a~hardwarové architektury detektorové sítě ATLAS~TPX a~také implementace řídícího softwaru ATLAS TPX serveru.
\end{description}