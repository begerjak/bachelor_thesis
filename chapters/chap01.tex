\addbibresource{reference.bib}

\chapter{Úvod}\label{chap01}
Tato bakalářská práce se zabývá návrhem a~implementací software pro ovládání a~energetickou kalibraci sítě hybridních částicových pixelových detektorů umístěných uvnitř experimentu ATLAS na urychlovači LHC v~CERN. Tento projekt je nazývá ATLAS TPX (viz kapitola \ref{atlas}).

Jelikož proces kalibrace je zcela nezávislý na následném řízení těchto detektorů, je software členěn na dvě nezávislé části, a~to na energetickou kalibraci a~řízení sítě  ATLAS TPX.

Kalibrační software umožňuje průchod procesem zpracování kalibračních dat, od sestavení spekter z~naměřených snímků, přes jejich analýzu a~vytvoření jednotlivých kalibračních bodů, až po sestavení kalibrační funkce pro jednotlivé pixely detektoru.

Řídící software sítě ATLAS TPX slouží pro nezávislé ovládání funkce těchto detektorů. Umožňuje nastavování různých parametrů detektorů, řízení akvizice snímků, vyčítání naměřených dat a~jejich ukládání, ev. jejich odesílání datovému serveru pro jejich hlubší analýzu a~uložení do perzistentní centrální CERNské databáze (tzv. \texttt{EOS}). Zároveň tento řídící software poskytuje \texttt{JSON REST API} pro své vzdálené řízení a~také pro poskytování stavových informací o teto sítí CERNu.


\section{Motivace}
Ionizující záření je spjato s~naším světem už od počátku jeho existence. Jeho studium začalo koncem 19. století a~pomáhá nám pochopit podstatu hmoty, její interakce s~prostředím a~další vlastnosti. Tyto poznatky našly své uplatnění v~mnoha oborech, jako například ve~zdravotnictví, defektoskopii, energetice a~v mnoha dalších. Spolu s~rostoucími znalostmi o ionizujícím záření a~s~technologickým pokrokem se rozvíjela i detekční technika, která za poslední století prodělala veliký posun. Od prvních bublinových komor, až po polovodičové pixelové detektory, kterými se tato práce zabývá. 
%todo
	
\section{Struktura dokumentu}
\begin{description}
	\item[Kapitola] \ref{det} - \nameref{det}:
		V této kapitole jsou představeny hybridní částicové pixelové detektory rodiny Medipix, jejich rozdělení, principy detekce, provozní módy a~další vlastnosti detektorů, relevantní k~této práci.
	\item[Kapitola] \ref{calib} - \nameref{calib}:
		Tato kapitola pojednává o metodách energetické kalibrace částicových pixelových detektorů rodiny Medipix, pracujících v~\texttt{Time-Over-Treshold} módu a~také je zde zmíněna implementace jedné z~těchto metod pro účely kalibrace detektorové sítě ATLAS TPX.
	\item[Kapitola] \ref{atlas} - \nameref{atlas}:
		V rámci této kapitoly je popsán návrh softwarové a~hardwarové architektury detektorové sítě ATLAS~TPX a~také implementace řídícího softwaru ATLAS TPX serveru.
\end{description}