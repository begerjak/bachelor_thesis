\addbibresource{reference.bib}

\chapter{Energetická kalibrace}\label{calib}
Tato kapitola pojednává o metodách pro energetickou kalibraci hybridních částicových pixelových detektorů, pracujících v Time-Over-Treshold módu a o implementaci jedné z nich pro účely kalibrace detektorů sítě ATLAS TPX.

%********************************************************************************
% Motivace
%********************************************************************************
\section{Motivace}
Hybridní částicové pixelové detektory typu Timepix (viz \ref{det:tim}), disponují módem TOT (Time-Over-Treshold), který lze využít pro měření energie. Jak již bylo uvedeno v kapitole \ref{det:mod}, když částice interaguje s pixelem, pracujícím v tomto módu, ve kterém zanechá část své energie, dojde k vygenerování napěťového pulzu. Když je velikost tohoto pulzu větší, než treshold zasaženého pixelu, tak dojde je spuštění čítače, který začne počítat hodinové cykly měřící frekvence a zastaví se tehdy, když klesne hodnota napětí nazpět, pod úroveň tresholdu. Po skončení akvizice je hodnota tohoto čítače rovna hodnotě TOT, která zhruba odpovídá deponované energii interagující částice. Vztah mezi TOT a energií je ale silně nelineární a je podmíněn různými elektronickými a fyzikálními vlastnostmi daného pixelu. Určení tohoto vztahu je předmětem kalibrace, o které tato kapitola bude pojednávat.

%********************************************************************************
% Přehled kalibračních metod
%********************************************************************************
\section{Přehled kalibračních metod}

\subsection{Kalibrace detektorů za použití rentgenového záření}
Metoda kalibrace pixelových detektorů pracujících v TOT módu \cite{Jakubek2011S262} spočívá v měření rentgenové fluorescence (viz \cite{Jakubek-radiography_and_charge_sharing}),
což je děj ke kterému dochází, když je nějaký materiál 
\footnote{Pro kalibraci se používají kovy, na příklad Am, In, Cu, Fe apod.}
ozařován rentgenovým zářením, které vyráží excitované elektrony z jeho atomů. Je-li vyražen elektron na nižší energetické úrovní, tak elektron z vyšší energetické úrovně deexcituje a obsadí jeho místo. Přebytečnou energii ztratí ve formě vyzářeného fotonu, který je následně detektorem detekován. 

Díky této fluorescenci je detektor pomocí různých mono-energetických zdrojů záření, jejichž energie je předem známa, postupně ozařován. V rámci tohoto měření je třeba pro každý zdroj pořídit velké množství snímků, ze kterých jsou následně vyfiltrovány jen tzv. \texttt{Single-Hit} události, což jsou události, při kterých interagující částice zasáhla jen jeden pixel. Tyto události jsou filtrovány z důvodu dosažení vyšší kvality kalibrace které je dosaženo potlačením zkreslení, způsobeným \texttt{Charge Sharing} efektem (viz \ref{det:mod}).

\begin{figure}[th]
	\begin{center}
		\includegraphics[width=14cm]{figures/calib_gerf.png}
		\caption{Spektrum TOT hodnot jednoho pixelu s proložením Gaussovou funkcí, sečtenou s Gaussovou chybovou funkcí (tzv. error funkce). Zdrojem rentgenové fluorescence byla měď.}
		\label{fig:calib:gerf}
	\end{center}
\end{figure}

Dalším krokem tohoto procesu je získání jednotlivých kalibračních bodů, udávající závislost mezi jednotlivými energiemi a TOT hodnotami. Toho je možné docílit vytvořením spekter pro jednotlivé pixely a zdroje záření. Na obrázku \ref{fig:calib:gerf} je příklad takového spektra pro jeden pixel detektoru a fluorescenčního záření z mědi. Na vodorovné ose tohoto spektra se nachází jednotlivé TOT hodnoty a na svislé pak jejich četnost ve všech snímcích. Z obrázku je patrné, že nejčetnější hodnotou TOT je zhruba hodnota $53$, které odpovídá energii fotonu, vyraženého z mědi, což je $5,9~keV$. Hodnotu TOT je ale třeba znát přesně, toho je docíleno proložením spektra funkcí \ref{eq:calib:gerf}, která vznikla z Gaussovy funkce, ke které z důvodu levé nesymetrie kvůli \texttt{Charge Sahring} efektu byla přičtena Gaussova chybová funkce. 

\begin{equation}\label{eq:calib:gerf}
	f_{GERF}(x) = \underbrace{Ae^{ -\frac{(x-\mu)^2}{2\sigma^2} }}_{\text{Gaussova funkce}} +
	\underbrace{ \frac{avg_{right} - avg_{left}}{\sigma\sqrt{2\pi}} \int_{-\infty}^t e^{ -\frac{(t-\mu)^2}{2\sigma^2} } + avg_{left}}_{\text{Gaussova chybová funkce}}
\end{equation}

Parametry funkce \ref{eq:calib:gerf} jsou následující:
\begin{itemize}
	\item $\mathbf{A}$ je amplituda.
	\item $\mathbf{\mu}$ je stření hodnota energie, kterou hledáme.
	\item $\mathbf{\sigma}$ udává rozptyl střední hodnoty energie a je možné ji vypočítat ze vzorce 
		$\sigma = \frac{2\sqrt{2ln_2}}{FWHM}$, kde $FWHM$\footnote{z angl. Full Width at Half Maximum} udává šířku gausiánu v polovině jeho výšky.
	\item $\mathbf{avg_{right}}$ (resp. $\mathbf{avg_{left}}$) je průměrná hodnoty spektra na pravém (resp. levém) úpatí gausiánu.
\end{itemize}
 
\begin{figure}[th]
	\begin{center}
		\includegraphics[width=13cm]{figures/calib_function.png}
		\caption{Kalibrační funkce (převzato z \cite{Jakubek2011S262})}
		\label{fig:calib:calib_function}
	\end{center}
\end{figure}

Z těchto kalibračních bodů je možné sestavit kalibrační funkci (viz vzorec \ref{eq:calib:calib_function}), udávající závislost mezi energií a TOT. Tato funkce vznikla složením hyperboly (popisující nelineární oblast nižších energií) a přímky (pro oblast s vyšší energií). Na obrázku \ref{fig:calib:calib_function} je znázorněn příklad této funkce.

\begin{equation}\label{eq:calib:calib_function}
	f_{calib}(x) = ax + b - \frac{c}{x-t}
\end{equation}

%\section{Implementace kalibrační metody pomocí ren}