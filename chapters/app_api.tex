\addbibresource{reference.bib}

\chapter{Metody REST API serveru}\label{app:api}
V této příloze naleznete popis všech REST API metod, poskytovaných \texttt{Atlas TPX serverem} a také použitých standardů.

% použito Java formátování, protože JSON nepodporuje komentáře
\begin{minted}[tabsize=4, frame=single, linenos]{java} 
{
	"data": {...},
	"error": {
		"errorMessage": "ok",
		"errorCode": 0
	},
	"success": true
}
\end{minted}
\sloppy
\begin{code}[h!]
	\caption{Základní struktura výstupních dat}
	\label{app:api:struktura_vystupnich_dat}
\end{code}

Vstupní data se liší dle metody, avšak základní struktura výstupních dat je pro všechny metody stejná - viz obr. \ref{app:api:struktura_vystupnich_dat}, kde objet \texttt{data} obsahuje výstupní data danné medoty (při dalším popisování vystupu jednotlivých metod, bude popisován jen obsah tohoto objektu), objekt \texttt{error} obsahuje kód chyby a její zprávu a boolean proměnná \texttt{success} udává, zda-li se zpracování příkazu zdařilo.

Url schéma jednotlivých metod je následující - \texttt{protokol://host:port/vx/skupina/metoda}, kde
\begin{description}
	\item[protokol] je použitý protokol (\texttt{HTTP/HTTPS}),
	\item[host] je doména, či IP adresa serveru,
	\item[port] - port serveru,
	\item[vx] je verze API, na které komunikace bude probíhat (současná verze je "v1"),
	\item[skupina a metoda] udávají cestu a název metody (u některých metod stačí zadat pouze skupinu).
\end{description}

\section{Použité obecné standardy}
\begin{itemize}
	\item Všechny zde popsané metody jsou založeny na HTTP metodě POST.
	\item Všechna předávaná data jsou ve formátu \texttt{JSON}. 
	\item Všechny časy a data jsou reprezentovány v milisekundách od 1.1.1970 (tzv. UNIX Time) a časové intervaly v sekundách, není-li specifikováno jinak.
	\item Server může vracet zprávy s následujícími HTTP kódy: 200 (OK), 400 (Bad Request), 404 (Not Found) a 500 (Internal Server Error).
\end{itemize}

\section{Získání stavových informací}
Pro získání stavových informací serveru a detektorů slouží metoda \texttt{/v1/info}. Pomocí jejích vstupních dat je možné nastavit vyloučení jednotlivých kategorií odpovědi. Jejich struktura vypadá následovně:

\begin{minted}[tabsize=4, frame=single, linenos]{java}
{
    "excludeDetectors" : false, // vyloučení přehledu detektorů
    "excludeDetectorSettings" : false, // vyloučení nastavení detektorů
    "excludeHwStatus" : false, // vyloučení hardwarového stavu detektorů
    "excludeDataServerStatus" : false // vyloučení stavu datového serveru
}
\end{minted}
\sloppy
\begin{code}[h!]
	\caption{Metoda \texttt{/v1/info} - vstupní JSON}
	\label{app:api:info-in}
\end{code}









