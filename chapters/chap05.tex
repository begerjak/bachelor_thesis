\addbibresource{reference.bib}

\chapter{Závěr}
Hlavním cílem této bakalářské práce byl vývoj software pro řízení sítě hybridních částicových pixelových detektorů (označované jako \texttt{ATLAS TPX}), operujících uvnitř ATLAS experimentu na LHC v~CERN - viz kapitola \ref{calib}. V~rámci této práce byl navržen a~implementován software, který pomocí multi-vláknového přístupu umožňuje současně ovládat všechny detektory této sítě, zejména pak řízení akvizice dat, vyčítání pořízených snímků, nastavování měřících parametrů a~také vyčítání a~uchovávání stavových informací detektorů. Řídící software rovněž poskytuje \texttt{JSON REST API} server pro vzdálené ovládání, díky kterému bude mimo jiné možné řízení detektorové sítě implementovat do CERNského systému pro řízení a~kontrolu detektorů (tzv. \texttt{DCS}). Další funkcionalitou tohoto software je zpracování a~ukládání pořízených dat z~detektorů, ty mohou být ukládány v~\texttt{Multiframe} formátu do předem definovaného úložiště nebo odesílány datovému serveru pomocí jeho \texttt{API}, který data dále zpracuje a~uloží do CERNského systému pro skladování dat (\texttt{EOS}).

Dalším cílem této práce byl vývoj nástrojů pro energetickou kalibrací sítě \texttt{ATLAS TPX}, pomocí metody popsané v~\cite{Jakubek2011S262}. Byl vyvinut univerzální software \ref{calib:sw}, který uživateli umožňuje průchod procesem zpracování dat pro energetickou kalibraci detektoru typu Timepix, pracujícího v~\texttt{Time-Over-Treshold} módu. V~rámci kalibračního procesu software ze spekter jednotlivých pixelů a~pro různé zdroje záření vytvoří jednotlivé kalibrační body, ze kterých pak sestaví kalibrační funkci (viz vzorec \ref{eq:calib:calib_function}), udávající vztah mezi energií a~TOT, pro každý pixel detektoru. Software rovněž nabízí nástroje pro ověření kvality vstupních dat, kvality detektoru, ověření kvality kalibrace a~také pro různé dodatečné úpravy výstupu kalibračního procesu, popsané v~\ref{calib:sw:post_process}.

V této práci byly rovněž shrnuty a~popsány vlastnosti hybridních částicových pixelových detektorů rodiny Medipix - viz kapitola \ref{det}.